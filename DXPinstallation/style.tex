\documentclass[letterpaper,11pt,pagesize=pdftex,openright,headings=twolinechapter,chapterprefix=true,oneside]{scrbook}

\usepackage{graphicx}
\usepackage{wrapfig}
\usepackage[table]{xcolor}
\usepackage{makeidx}
\usepackage{fontspec}
\usepackage{longtable}
\usepackage{nameref}

\setmainfont[Ligatures={Common,TeX}, Numbers={OldStyle}]{Nimbus Sans L}
%\KOMAoptions{DIV=last}   % goofy KOMA requirement whenever you change leading
\addtokomafont{footnote}{\footnotesize\fontspec[Ligatures=TeX]{Alegreya Sans}}
\deffootnote{1.5em}{1em}{% modified example from page 83
  \makebox[1.5em][l]{\textsuperscript{\thefootnotemark}}}


\makeindex

\usepackage[inner=1in,outer=1in,top=1.0in, bottom=1.0in, footnotesep=.4in]{geometry}
\usepackage{changepage, hyperref, graphicx}

\newenvironment{myindent}{\begin{adjustwidth}{2.0em}{}}{\end{adjustwidth}}
\newcommand*{\thead}[1]{\multicolumn{1}{|c|}{\bfseries #1}}

%Page size settings
\pdfpagewidth=\paperwidth
\pdfpageheight=\paperheight


\hyphenation{so-lvers hun-ches Golds-worthy pre-dic-ta-ble com-pu-te-ri-sa-tion self-in-te-re-sted tech-ni-qu-es tre-men-do-us}


\setlength{\emergencystretch}{15pt}


%control whether extra vertical space is distributed between paragraphs
%either raggedbottom or flushbottom
\raggedbottom

\usepackage{hyperref}
\hypersetup{
    colorlinks,
    citecolor=black,
    filecolor=black,
    linkcolor=purple,
    urlcolor=blue
}


\usepackage{scrhack}
\usepackage{lineno}
\usepackage{listings}

\lstset{ %
  backgroundcolor=\color{white},   % choose the background color; you must add \usepackage{color} or \usepackage{xcolor}
  basicstyle=\ttfamily\footnotesize,% the size of the fonts that are used for the code
  breakatwhitespace=false,         % sets if automatic breaks should only happen at whitespace
  breaklines=true,                 % sets automatic line breaking
  captionpos=b,                    % sets the caption-position to bottom
  commentstyle=\color{green},      % comment style
  deletekeywords={...},            % if you want to delete keywords from the given language
  escapeinside={\%*}{*)},          % if you want to add LaTeX within your code
  extendedchars=true,              % lets you use non-ASCII characters; for 8-bits encodings only, does not work with UTF-8
  frame=single,                    % adds a frame around the code
  keepspaces=true,                 % keeps spaces in text, useful for keeping indentation of code (possibly needs columns=flexible)
  keywordstyle=\color{blue},       % keyword style
  language=Java,                   % the language of the code
  otherkeywords={*,...},           % if you want to add more keywords to the set
  numbers=right,                   % where to put the line-numbers; possible values are (none, left, right)
  numbersep=5pt,                   % how far the line-numbers are from the code
  numberstyle=\tiny\color{gray},   % the style that is used for the line-numbers
  rulecolor=\color{black},         % if not set, the frame-color may be changed on line-breaks within not-black text (e.g. comments (green here))
  showspaces=false,                % show spaces everywhere adding particular underscores; it overrides 'showstringspaces'
  showstringspaces=false,          % underline spaces within strings only
  showtabs=false,                  % show tabs within strings adding particular underscores
  stepnumber=5,                    % the step between two line-numbers. If it's 1, each line will be numbered
  stringstyle=\color{blue},        % string literal style
  tabsize=2,                       % sets default tabsize to 2 spaces
  title=\lstname                   % show the filename of files included with \lstinputlisting; also try caption instead of title
}

\newenvironment{urlboxstyle}
    {\vspace{.15in}\noindent %
	\arrayrulecolor{purple} %
    \begin{ttfamily}
	\begin{tabular}{|p{0.9\textwidth}|}
    \hline
    }
    {
    \\\hline
    \end{tabular}
    \end{ttfamily}
	\arrayrulecolor{black} %
	\vspace{.15in}
    }


\newcommand{\imagehere}[1]{%
    \vspace{.25in}%
    \begin{center}%
      \includegraphics[width=\textwidth,height=3in,keepaspectratio=true]{prt-img/#1}%
    \end{center}%
    \vspace{.25in}%
}

%TABLES
%make more space between lines
\renewcommand{\arraystretch}{1.3}


\newcommand{\librarysection}[1]{%
    \index{#1}%
    \chapter{#1}%
}

\newcommand{\librarycommand}[1]{%
    \newpage%
    \index{#1}%
    \section{#1}%
}

\newenvironment{inputblock}
    {
    Input:
    \begin{itemize}
    }
    {
    \end{itemize}
    }

\newenvironment{outputblock}
    {
    Output:
    \begin{itemize}
    }
    {
    \end{itemize}
    }


%this macro creates the paragraphs with a bold title at the beginning
%a little extra space before, and no indent.  This should be used only
%at the beginning of a paragraph, and the paramter is the paragraph title.
%Do not put punctuation after the title.  The following sentence should
%start with a capital letter.
\newcommand{\paratitle}[1]{\vspace{3mm}\noindent\textbf{#1} \hspace{2mm}}


%\usepackage{draftwatermark}
%\SetWatermarkText{Preview Copy}
%\SetWatermarkScale{0.5}


\newenvironment{fieldlist} %
    {\begin{center} %
    \begin{longtable}{@{}p{0.2\textwidth}p{0.8\textwidth}@{}} %
    \hline %
    } %
    { %
    \end{longtable} %
    \end{center} %
    }

\newcommand{\fieldlistitem}[2] {%
    \emph{#1} & #2\\ \hline %
}

\newcommand{\datalistitem}[2] {%
    \textbf{#1} & #2\\ \hline %
}

\newenvironment{glossarylist}%
   {\begin{list}{}{\setlength\labelwidth{0pt}%
                   \setlength\itemindent{-\leftmargin}%
                   \let\makelabel\descriptionlabel}}%
   {\end{list}}

\newcommand{\chapterauthor}[1]{
    \begin{center}
    {\fontspec{Verdana}\fontsize{14}{16}\bfseries #1}
    \end{center}
}
\newcommand{\chapterabstract}[1]{
    \begin{quote}
    \emph{#1}
    \end{quote}
}


\newcommand{\crossref}[1] {section~\ref{#1} \nameref{#1} on page~\pageref{#1}}
